\batchmode
\makeatletter
\def\input@path{{/var/tmp/paciorek/paleon/biomass_product_paper//}}
\makeatother
\documentclass[12pt]{article}\usepackage[]{graphicx}\usepackage[]{color}
%% maxwidth is the original width if it is less than linewidth
%% otherwise use linewidth (to make sure the graphics do not exceed the margin)
\makeatletter
\def\maxwidth{ %
  \ifdim\Gin@nat@width>\linewidth
    \linewidth
  \else
    \Gin@nat@width
  \fi
}
\makeatother

\definecolor{fgcolor}{rgb}{0.345, 0.345, 0.345}
\newcommand{\hlnum}[1]{\textcolor[rgb]{0.686,0.059,0.569}{#1}}%
\newcommand{\hlstr}[1]{\textcolor[rgb]{0.192,0.494,0.8}{#1}}%
\newcommand{\hlcom}[1]{\textcolor[rgb]{0.678,0.584,0.686}{\textit{#1}}}%
\newcommand{\hlopt}[1]{\textcolor[rgb]{0,0,0}{#1}}%
\newcommand{\hlstd}[1]{\textcolor[rgb]{0.345,0.345,0.345}{#1}}%
\newcommand{\hlkwa}[1]{\textcolor[rgb]{0.161,0.373,0.58}{\textbf{#1}}}%
\newcommand{\hlkwb}[1]{\textcolor[rgb]{0.69,0.353,0.396}{#1}}%
\newcommand{\hlkwc}[1]{\textcolor[rgb]{0.333,0.667,0.333}{#1}}%
\newcommand{\hlkwd}[1]{\textcolor[rgb]{0.737,0.353,0.396}{\textbf{#1}}}%
\let\hlipl\hlkwb

\usepackage{framed}
\makeatletter
\newenvironment{kframe}{%
 \def\at@end@of@kframe{}%
 \ifinner\ifhmode%
  \def\at@end@of@kframe{\end{minipage}}%
  \begin{minipage}{\columnwidth}%
 \fi\fi%
 \def\FrameCommand##1{\hskip\@totalleftmargin \hskip-\fboxsep
 \colorbox{shadecolor}{##1}\hskip-\fboxsep
     % There is no \\@totalrightmargin, so:
     \hskip-\linewidth \hskip-\@totalleftmargin \hskip\columnwidth}%
 \MakeFramed {\advance\hsize-\width
   \@totalleftmargin\z@ \linewidth\hsize
   \@setminipage}}%
 {\par\unskip\endMakeFramed%
 \at@end@of@kframe}
\makeatother

\definecolor{shadecolor}{rgb}{.97, .97, .97}
\definecolor{messagecolor}{rgb}{0, 0, 0}
\definecolor{warningcolor}{rgb}{1, 0, 1}
\definecolor{errorcolor}{rgb}{1, 0, 0}
\newenvironment{knitrout}{}{} % an empty environment to be redefined in TeX

\usepackage{alltt}
\usepackage[T1]{fontenc}
\usepackage[latin9]{inputenc}
\usepackage{geometry}
\geometry{verbose,tmargin=1in,bmargin=1in,lmargin=1in,rmargin=1in}
\usepackage{color}
\usepackage[authoryear]{natbib}
\usepackage[unicode=true]
 {hyperref}

\makeatletter
%%%%%%%%%%%%%%%%%%%%%%%%%%%%%% Textclass specific LaTeX commands.
\newcommand{\lyxaddress}[1]{
\par {\raggedright #1
\vspace{1.4em}
\noindent\par}
}

%%%%%%%%%%%%%%%%%%%%%%%%%%%%%% User specified LaTeX commands.
\usepackage{times}
\usepackage{graphics,natbib}

\newcommand{\matern}{Mat\'{e}rn }
\newcommand{\N}{\mathcal{N}}

\makeatother
\IfFileExists{upquote.sty}{\usepackage{upquote}}{}
\begin{document}


\title{Statistically-estimated biomass for the northeastern United States:
modern and \\
Euro-American settlement-era}


\author{Christopher J. Paciorek\textsuperscript{1,{*}}, \\
{[}in no particular order{]} Simon J. Goring, Charles V. Cogbill,
Mike Dietze, John W. Williams, Andria Dawson, Jody A. Peters, Jun
Zhu, Xiaoping Feng, and Jason S. McLachlan\textsuperscript{7}}

\maketitle

\lyxaddress{\textsuperscript{1}Department of Statistics, University of California,
Berkeley, California, USA}


\lyxaddress{\textsuperscript{2}fill in other authors: check in with David Mladenoff,
Kelly Heilman, Dave Moore}


\lyxaddress{\textsuperscript{{*}}Corresponding author; E-mail: paciorek@stat.berkeley.edu}


\lyxaddress{\newpage}
\begin{quote}


NOTE: much of this text is directly from the composition paper and
needs substantial rewording and addition of analogous content for
FIA.
\end{quote}

\begin{abstract}
{[}Edit to transform to biomass not composition{]} We present a gridded
8 km-resolution data product of the estimated biomass of tree taxa
at the time of Euro-American settlement of the northeastern United
States and the statistical methodology used to produce the product
from trees recorded by land surveyors. . The data come from settlement-era
public survey records that are transcribed and then aggregated spatially,
giving count data. The domain is divided into two regions, eastern
(Maine to Ohio) and midwestern (Indiana to Minnesota). Public Land
Survey point data in the midwestern region (ca. 0.8-km resolution)
are aggregated to a regular 8 km grid, while data in the eastern region,
from Town Proprietor Surveys, are aggregated at the township level
in irregularly-shaped local administrative units. The product is based
on a Bayesian statistical model fit to the count data that estimates
composition on a regular 8 km grid across the entire domain. The statistical
model is designed to handle data from both the regular grid and the
irregularly-shaped townships and allows us to estimate composition
at locations with no data and to smooth over noise caused by limited
counts in locations with data. Critically, the model also allows us
to quantify uncertainty in our composition estimates, making the product
suitable for applications employing data assimilation. We expect this
data product to be useful for understanding the state of vegetation
in the northeastern United States prior to large-scale Euro-American
settlement. In addition to specific regional questions, the data product
can also serve as a baseline against which to investigate how forests
and ecosystems change after intensive settlement. The data product
is being made available at the NIS data portal as version 1.0.

Keywords: biogeography, biomass, old-growth forests, spatial modeling,
Bayesian statistical model, vegetation mapping
\end{abstract}

\section{Introduction}

Historical datasets provide critical context to understand forest
ecology. They allow researchers to define `baseline' conditions for
conservation management, to understand ecosystem processes at decadal
and centennial scales, to track forest responses to shifting climates,
and, particularly in regions with widespread land use change, to understand
the extent to which forests after conversion and regeneration differ
from the original forest cover.

Euro-American settlement and subsequent land use change occurred in
a time-transient fashion across North America and were accompanied
by land surveys needed to demarcate land for land tenure and use.
Various systems were used by surveyors to locate legal boundary markers,
usually by recording and marking trees adjacent to survey markers.
These data provide vegetation information that can be mapped and used
quantitatively to represent the period of settlement. Early surveys
(from 1620 until 1825) in the northeastern United States provide spatially-aggregated
data at the township level \citep{Cogb:etal:2002,thompson2013four},
with typical township size on the order of 200 $\mbox{km}^{2}$ and
no information about the locations of individual trees; we refer to
these as the Town Proprietor Survey (TPS). Later surveys after the
establishment of the U.S. Public Land Survey System (PLS) by the General
Land Office (GLO) provide point-level data along a regular grid, with
one-half mile (800 m) spacing, for Ohio and westward during the period
1785 to 1907 \citep{bourdo1956review,pattison1957beginnings,schulte2001original,goring2015composition}.
At each point 2-4 trees were identified, and the common name, diameter
at breast height, and distance and bearing from the point were recorded.
Survey instructions during the PLS varied through time and by point
type. Accounting for this variation requires data screening to maximize
consistency among points and the application of spatially-varying
correction factors \citet{goring2015composition} to accurately assess
tree stem density, basal area and biomass from the early settlement
records, but the impact on composition estimates is limited \citep{liu2011broadscale}.
Surveyors sometimes used ambiguous common names, which requires matching
to scientific names and standardization \citep{mladenoff2002narrowing,goring2015composition}.

Logging, agriculture, and land abandonment have left an indelible
mark on forests in the northeastern United States \citep{foster1998land,rhemtulla2009legacies,thompson2013four,goring2015composition}.
However most studies have assessed these effects in individual states
or smaller domains \citep{friedman2005regional,rhemtulla2009historical}
and with various spatial resolutions, from townships (36 square miles)
to forest zones of hundreds or thousands of square miles. \citet{goring2015composition}
provide a new dataset of forest composition, biomass, and stem density
based on PLS data for the upper Midwest that is resolved to an 8 km
by 8 km grid cell scale, providing broad spatial coverage at a spatial
scale that can be compared to modern forests using Forest Inventory
and Analysis products \citep{gray2012forest}. Combined with additional,
coarsely-sampled PLS data from Illinois and Indiana, newly-digitized
data from southern Michigan, and with the TPS data, this gives us
raw data for much of the northeastern United States. However, there
are several limitations of using the raw data that can be alleviated
by the use of a statistical model to develop a statistically-estimated
data product. First, the PLS and TPS data only provide estimates of
within-cell variance that do not account for information from nearby
locations. Second, there are data gaps: the available digitized data
from Illinois and Indiana represent a small fraction of those states,
and missing townships are common in the TPS data. Third, the TPS and
PLS data have fundamentally different sampling design and spatial
resolution. Our statistical model allows us to provide a spatially-complete
data product of settlement-era tree composition for a common 8 km
grid with uncertainty across the northeastern U.S.

In Section \ref{sec:Data} we describe the data sources, while Section
\ref{sec:Statistical-model} describes our statistical models. In
Section \ref{sec:Model-comparison} we quantitatively compare competing
statistical specifications, and in Section \ref{sec:Data-product}
we describe the final data product. In Section \ref{sec:Discussion}
we discuss \textcolor{red}{the uncertainties estimated by }and the
limitations of the statistical model, and we list related data products
under development.




\section{Data\label{sec:Data}}


\subsection{Public Land Survey}

The raw data were obtained from land division survey records collated
and digitized from across the northeastern U.S. by a number of researchers
(Fig. \ref{fig:Spatial-domain}). For the states of Minnesota, Wisconsin,
Illinois, Indiana, and Michigan (the midwestern subdomain), digitized
data are available at PLS survey point locations and have been aggregated
to a regular 8 km grid in the Albers projection. (Note that for Indiana
and Illinois, at the moment trees are associated with township centroids
and then assigned to 8 km grid cells based on the centroid, but in
the near future we will have point locations available for each tree.)
For the states of Ohio, Pennsylvania, New Jersey, New York and the
six New England states (the eastern subdomain), data are aggregated
at the township level. \textcolor{red}{We make predictions for all
of the states listed above; these constitute our core domain. }There
are also data from a single township in Quebec and a single township
in northern Delaware\textcolor{red}{; these data help inform predictions
in nearby locations within our core domain, but predictions are not
made for Quebec or Delaware}. Digitization of PLS data in Minnesota,
Wisconsin and Michigan is essentially complete, with PLS data for
nearly all 8 km grid cells, but data in Illinois and Indiana represent
a sample of the full set of grid cells, with survey record transcription
ongoing. Data for the eastern states are available for a subset of
the full set of townships covering the domain; the TPS data for some
townships were lost, incomplete, or have not been located \citep{Cogb:etal:2002}. 

\begin{figure}
\label{fig:domain}

\caption{Spatial domain of the northeastern United States, with locations with
data shown in gray. Locations are grid cells in midwestern portion
and townships in eastern portion. In addition to locations without
data being indicated in white, grid cells completely covered in water
are white (e.g., a few locations in the northwestern portion of the
domain in the states of Minnesota and Wisconsin).\label{fig:Spatial-domain}}
\end{figure}


Note that surveys occurred over a period of more than 200 years as
European colonists (before U.S. independence) and the United States
settled what is now the northeastern and midwestern United States.
Our estimates are for the period of settlement represented by the
survey data and therefore are time-transgressive; they do not represent
any single point in time across the domain, but rather the state of
the landscape at the time just prior to widespread Euro-American settlement
and land use \citep{Whit:1996,Cogb:etal:2002}. These forest composition
datasets do include the effects of Native American land use and early
Euro-American settlement activities \citep[e.g.,][]{Blac:etal:2006},
but it is likely that the imprint of this earlier land use is highly
concentrated rather than spatially extensive \citep{munoz2014defining}.

Extensive details on the upper Midwest (Minnesota, Wisconsin, Michigan)
data and processing steps are available \cite{goring2015composition};
key elements include the use of only corner points, the use of only
the two closest trees at each corner point, spatially-varying correction
factors for sampling effort, and a standardized taxonomy table. The
lower Midwest (Illinois, Indiana) data were purchased from the Indiana
State Archives (Indiana) and Hubtack Document Resources (hubtack.com;
Illinois) and processed using similar steps as for the upper Midwest
data. Digitization of the Illinois and Indiana data is still underway,
so many grid cells contained no data at the time the statistical model
was fit. Note that the number of trees per grid cell varies depending
on the number of survey points in a cell, with an average of 124 trees
per cell. The gridded data at the 8 km resolution for the midwest
subdomain are available through the NIS data portal \textcolor{red}{\citep{Gori:etal:data:2016}}.
The TPS data were compiled by C.V. Cogbill from a myriad of archival
sources representing land division surveys conducted in connection
with local settlement\textcolor{red}{{} and are available through the
NIS data portal \citep{Cogb:dataNE:2016,Cogb:dataOH:2016}}. 

The aggregation into taxonomic groups is primarily at the genus level
but is at the species level in some cases of monospecific genera.
We model the following 22 taxa plus an ``other hardwood'' category:
Atlantic white cedar (\emph{Chamaecyparis thyoides}), Ash (\emph{Fraxinus
spp.}), Basswood (\emph{Tilia americana}), Beech (\emph{Fagus grandifolia)},
Birch (\emph{Betula spp.}), Black gum/sweet gum (\emph{Nyssa sylvatica}
and \emph{Liquidambar styraciflua}), Cedar/juniper (\emph{Juniperus
virginiana} and \emph{Thuja occidentalis}), Cherry (\emph{Prunus spp.}),
Chestnut (\emph{Castanea dentata}), Dogwood (\emph{Cornus spp.}),
Elm (\emph{Ulmus spp.}), Fir (\emph{Abies balsamea}), Hemlock (\emph{Tsuga
canadensis}), Hickory (\emph{Carya spp.}), Ironwood (\emph{Carpinus
caroliniana} and \emph{Ostrya virginiana}), Maple (\emph{Acer spp.}),
Oak (\emph{Quercus spp.}), Pine (\emph{Pinus spp.}), Poplar/tulip
poplar (\emph{Populus spp.}~and \emph{Liriodendron tulipifera}),
Spruce (\emph{Picea spp.}), Tamarack (\emph{Larix laricina}), and
Walnut (\emph{Juglans spp.}). Note that in several cases (black gum/sweet
gum, ironwood, poplar/tulip poplar, cedar/juniper), because of ambiguity
in the common tree names used by surveyors, a group represents trees
from different genera or even families. For the midwestern subdomain
we do not fit statistical models for Atlantic white cedar and chestnut
as these species have 0 and 7 trees present, respectively. The taxa
grouped into the other hardwood category are those for which fewer
than roughly 2000 trees were present in the dataset; however, we include
Atlantic white cedar explicitly despite it only having 336 trees in
the dataset because of specific ecological interest in Atlantic white
cedar wetlands. 

Diameters are only recorded in the PLS data. Although surveyors avoided
using small trees, there was no consistent lower diameter limit. The
PLS data generally represent trees greater than 8 inches (\textasciitilde{}20
cm) diameter at breast height (dbh), but with some trees as small
as 1 inch dbh (smaller trees were much more common in far northern
Minnesota). TPS data have no information about dbh, but the trees
were large enough to blaze and are presumed to be relatively large
trees useful for marking property boundaries.

There are approximately 860,000 trees in the midwestern subdomain
and 420,000 trees in the eastern subdomain. In the midwestern subdomain,
oak is the most common taxon and pine the second most common, while
in the eastern subdomain oak is the most common and beech the second
most common.

Our domain is a rectangle covering all of the states using a metric
Albers (Great Lakes and St. Lawrence) projection (PROJ4: EPSG:3175),
with the rectangle split into 8 km cells, arranged in a 296 by 180
grid of cells, with the centroid of the cell in the southwest corner
located at (-71000 m, 58000 m). For the midwestern subdomain we use
the western-most 146 by 180 grid of cells\textcolor{red}{{} when fitting
the statistical models}. For the eastern subdomain we use the eastern-most
180 by 180 grid of cells and then omit 23 rows of cells in the north
and 17 rows of cells in the south, as these grid cells are outside
of the states containing data.

Data processing steps:
\begin{itemize}
\item See Goring et al. for processing of upper Midwest data; similar steps
for IL/IN/So MI; result is tree dbh for trees at each PLS point
\item See Goring et al. for methods to estimate density at each point based
on distance to survey point and various correction factors (cite Cogbill
in press)
\item scale dbh to biomass per tree
\item scale by point density to get biomass per unit area for each taxon
present at a point
\item aggregate biomass within grid cells:

\begin{itemize}
\item proportion of PLS points within each cell containing a given taxon
\item average biomass per taxon across occupied points
\end{itemize}
\end{itemize}

\subsection{Forest Inventory and Analysis}

We use data from the most recent data collection at each FIA plot
in the focal states, with no data prior to 1999 used because of changes
in FIA methodology at that point in time, which make it difficult
to be sure we are not using multiple surveys from a single plot. Also
this limits the degree of time transgressiveness. 

Data processing steps:
\begin{itemize}
\item download from DataMart
\item based on MOU, assign plots to PalEON grid cells
\item scale dbh to biomass per tree
\item compute total biomass per taxon within each FIA plot
\item aggregate biomass within grid cells:

\begin{itemize}
\item proportion of FIA plots within each cell containing a given taxon
\item average biomass per taxon across occupied cells
\end{itemize}
\end{itemize}

\subsection{Allometric scaling}

{[}tradeoff of taxon resolution vs. ability to use random site effects
difficulty of capturing spatial patterns in allometry, so we capture
variability in individual tree allometry but have spatial biases and
overall bias from uncertainty in allometry given limited spatial info
and limited data in allometry papers use of component 6 {]}


\section{Statistical model\label{sec:Statistical-model}}

The major challenge of modeling biomass data is that biomass is positive-valued
but continuous, for which limited statistical distributions are available.

In early efforts we considered a Tweedie model but encountered computational
difficulties in model convergence and concerns about how well the
model fit. Given this we developed a two-stage model to address the
challenge of zero inflation in non-negatively valued distributions.

There are many zero-inflated models in the statistical literature,
most focusing on count or proportional data {[}need some lit review{]}.
{[}look for zero-inflated continuous lit{]}. Our model was motivated
by the biological insight that local conditions may prevent a taxon
from occurring in an area even though the taxon may be present at
high density nearby. Thus we combine a model for ``potential biomass'',
which reflects the large-spatial-scale patterns in biomass with a
model for ``occupancy'', which reflects the propensity for a given
forest stand to contain the taxon. This model allows for zero inflation
with a small value for the occupancy model in a given location.

Let $N(s)$ be the number of PLS sample points or FIA plots in grid
cell $s$. Let $n_{p}(s)$ be the number of points in the cell that
have one or more trees of taxon $p$. Let $\bar{Y}_{p}(s)$ be the
average biomass for taxon $p$ calculated ONLY from the $n_{p}(s)$
points at which the taxon is present. In other words, $\bar{Y}_{p}(s)=\frac{1}{n_{p}(s)}\sum_{i=1}^{n_{p}(s)}Y_{ip}(s)$
where $i$ indexes sites within cell $s$, $i=1,\ldots,n_{p}(s)$.
Let $m_{p}(s)$ be the potential (log) biomass process, evaluated
at grid cell $s$, and $\theta_{p}(s)$ be the occupancy patial process.
The biomass in a cell can then be calculated as $b_{p}(s)=\theta_{p}(s)\exp(m_{p}(s))$,
namely weighting the average biomass in \textquotedblleft occupied
patches\textquotedblright{} by the proportion of patches that have
the taxon. 

Let\textquoteright s first consider the occupancy model. The likelihood
is binomial, $n_{p}(s)\sim\mbox{Bin}(N(s),\theta_{p}(s)$. Note that
the occupancy model represents the occupancy of patches within a grid
cell, and that $\sum_{p}\theta_{p}(s)>1$ because two taxa will often
\textquotedblleft occupy\textquotedblright{} the same patch since
most PLS points have two trees. Next consider the (log) biomass process.
The likelihood is taken to be normal, $\mbox{\ensuremath{\log}}\bar{Y}_{p}(s)\sim\mbox{N}(m_{p}(s),\frac{\sigma_{p}^{2}}{n_{p}(s)})$.
Note that this likelihood accounts for heteroscedasticity related
to the number of points at which the taxon is observed (not the number
of PLS points in the cell). 

This two-stage model is able to account for structural zeros (the
taxon is not present because local conditions prevent it) and the
resulting zero inflation through the occupancy model while also able
to capture the smooth larger-scale variation in biomass and the differential
amounts of information in the face of the large number of zeros and
different numbers of sampling points in each grid cell.

Note that $m_{p}(s)$ is likely to be quite smooth spatially, at least
for the PLS data, because when a patch is occupied by a given taxon,
the tree is likely to be of adult size, regardless of whether the
tree is common in the grid cell. So most of the spatial variation
in biomass may be determined by variability in occupancy. The potential
biomass is meant to correct for the fact that density and tree size
may vary somewhat, but probably not drastically, across the domain.

This model is fit to both PLS and FIA data. For PLS, we have a large
number of points in each grid cell, while for FIA we have a small
(1-XX FIA plots per cell) number. 

We fit the two component models using the penalized splines to model
the spatial variation, with the fitting done by the numerically robust
generalized additive modeling (GAM) methodology implemented in the
R package mgcv. 

{[}discuss possible oversmoothing{]}

As discussed in Wood (XXXX), one can derive a quasi-Bayesian approach
and simulate draws from the quasi-posterior as follows: xxxxx. 

We combined draws from the occupancy and potential biomass processes
to produce biomass draws for each taxon and for total biomass. 

Note that one major drawback of this methodology is that the individual
taxon estimates are not constrained to add to a reasonable total biomass
because the taxa are fit individually. Further, as was the case in
our related modeling of composition, we do not capture correlations
between taxa in part to reduce computational bottlenecks and in part 

We fit a Bayesian statistical model to the data, with two primary
goals:
\begin{enumerate}
\item To estimate composition on a regular grid across the entire domain,
filling gaps where no data are available, and
\item To quantify uncertainty in composition at all locations. Even in grid
cells and townships with data, we wish to quantify uncertainty because
the empirical proportions represent estimates of the true proportions
that could be calculated using the full population of all the trees
in a grid cell or township.
\end{enumerate}
\textcolor{red}{At a high level, the statistical model estimates composition
across the domain, even in locations with sparse or no data, by combining
the raw composition data with the assumption that composition varies
in a smooth spatial fashion across the domain. The information in
the data is quantified by the data model, also known as the likelihood.
The assumption of smoothness is built into the model by representing
the true unknown spatially-varying composition using a statistical
spatial process representation that induces smoothing of estimates
across nearby locations. This spatial process representation is a
form of prior distribution and is a function of model parameters called
hyperparameters that determine the correlation structure of the process
and are also estimated based on the data. }

The result of fitting the Bayesian model via Markov chain Monte Carlo
(MCMC) is a set of representative samples from the posterior distribution
for the composition in the 23 taxonomic groupings at each of the grid
cells. These samples are the data product (described further in the
Section \ref{sec:Data-product}) and can be used in subsequent analyses.
The mean and standard deviation of the samples for each pair of cell
and taxon represent our best estimate (i.e., prediction) of composition
and a Bayesian ``standard error'' quantifying the uncertainty in
the estimate. 

\textcolor{red}{In the remainder of this section we provide the technical
specification of the model and of the computations involved in fitting
the model.}


\subsection{Computation}


\subsection{Assessment of smoothing}

{[}metric for CV should be between log(biomass) and biomass given
that we don't care about very small changes in low biomass but also
don't want to overemphasize errors in large biomass - show both in
terms of MSE for biomass and log(biomass)? also consider prediction
coverage{]}

{[}do in terms of biomass not (a) presence and (b) biomass when non-zero
given the two-step model is just a device to handle data distribution.
May need to have bivariate grid of gamma values for both models or
it may be that focusing on potential biomass part is enough {]}


\section{Scientific Results}


\section{Data product\label{sec:Data-product}}

The final data product is a dataset that contains 250 posterior samples
of the proportions of each of the 23 tree taxa at each grid cell in
the states in our domain of the northeastern United States.

For this final data product, we ran the model using the CAR specification
\textcolor{red}{with all of the data (including the data held out
in the model comparison analyses)} for 150,000 iterations with the
same burn-in and subsampling details as described in Section \ref{sec:Model-comparison}.
Based on graphical checks and calculation of effective sample size
values, mixing was generally reasonable, but for some of the hyperparameters
was relatively slow, particularly for less common taxa. Despite this,
mixing for the variables of substantive interest -- the proportions
-- was good, with effective sample sizes for the final product generally
near 250.

Maps of estimated composition for the full domain for several taxa
of substantive interest illustrate the results, contrasting the raw
data proportions, the posterior means, and posterior standard deviations
as pointwise estimates of uncertainty (Fig. \ref{fig:select_maps}).
We also present the posterior means for all 23 taxa (Fig. \ref{fig:all_predictions}).

\begin{figure}
\hspace{4mm}\hspace{4mm}\hspace{3.5mm}

\caption{Empirical proportions from raw data (column 1), predictions in the
form of posterior means (column 2) and uncertainty estimates in the
form of posterior standard deviations -- representing standard errors
of prediction (column 3) for select taxa. \label{fig:select_maps}}
\end{figure}


\begin{figure}
\caption{Predictions (posterior means) for all taxa over the entire domain.\label{fig:all_predictions}}


\end{figure}


The data product is publicly available at the \textcolor{red}{NIS
Data Portal }under the CC BY 4.0 license as version 1.0 as of January
2016 \citep{paci:etal:data:2016}. The product is in the form of a
netCDF-4 file, with dimensions x-coordinate, y-coordinate, and MCMC
iteration. There is one variable per taxon. In addition, dynamic visualizations
of the product using the Shiny tool are available at \href{https://www3.nd.edu/~paleolab/paleonproject/maps}{https://www3.nd.edu/$\sim$paleolab/paleonproject/maps}.
The PalEON Project (in particular the first author) will continue
to maintain this product, releasing new versions as additional data
in Illinois, Indiana and Ohio are digitized. Note that digitization
of data from Illinois and Indiana is ongoing, and digitization of
additional data from Ohio is planned as well. As a result, at some
point we expect to have complete data for the midwestern half of the
domain. 


\section{Discussion\label{sec:Discussion}}

\textcolor{red}{A key advance of this work over prior reconstructions
of settlement-era vegetation lies in the estimates of uncertainty
across the spatial domain. These estimates of uncertainty include
the sampling uncertainty within grid cells (as do the within-grid
cell estimates of uncertainty available from the raw proportions),
but, because this is a spatial model, predictions and their associated
uncertainty estimates are also informed by the information content
of nearby cells. The maps of standard errors across species (Fig.
\ref{fig:select_maps}, third column) highlight the advantages of
this approach in areas of high data coverage (Minnesota, Wisconsin,
Michigan) and in areas of sparse coverage (e.g., Illinois, Indiana,
parts of Ohio). Where there are not large gaps in the data, the model
provides low and fairly smooth estimates of uncertainty. Uncertainty
is generally higher in the eastern subdomain than in the areas of
the midwestern subdomain with high data coverage because of missing
townships and lower sampling density even in townships with data.
In areas of sparse coverage and in areas with low tree density (e.g.,
southwestern Minnesota), the standard error of our estimates increases
appropriately. Nevetheless, these uncertainties surround reasonable
estimates of trends in composition. For example, the model does a
good job of capturing the oak ecotone in Indiana and Illinois, representing
a shift from oak savannas and woodlands to closed mesic forests (Fig.
\ref{fig:select_maps}). Experiment 1 showed that both models predicted
composition at cells with no data reasonably well, mimicking the case
of sparsely sampled data and giving confidence in the broad spatial
patterns predicted in more poorly sampled regions, particularly those
with regular, but sparse sampling that mimic the experiment (Illinois
and Indiana, but not Ohio). The apparent blockiness of uncertainty
estimates in a few places such as Ohio is caused by spatial gaps and
variations in sampling resolution. Absolute uncertainty generally
increases with abundance for all taxa (Fig. \ref{fig:select_maps},
column 3).}




\section*{Author contributions}

CJP, and {[}JZ, XF{]} developed the statistical model and code. CJP
carried out the model comparison and created the data product. CJP
and CVC wrote the paper with feedback and editing from SJG, etc..
SJG, JAP, CVC, DJM, JSM, and JWW led the processing and analysis of
the PLS and TPS data and assisted with interpretation of results. 


\section*{Acknowledgments}

The authors are deeply indebted to all of the researchers over the
years who have preserved, collected, and digitized survey records,
in particular John Burk, Jim Dyer, Peter Marks, Robert McIntosh, Ed
Schools, Ted Sickley, Ronald Stuckey, and the Ohio Biological Survey.
We thank Madeline Ruid, Benjamin Seliger, Morgan Ripp and Daniel Handel
for processing of the southern Michigan data. Indiana and Illinois
data were made possible through the hard work of many Notre Dame undergraduates
in the McLachlan lab. This work was carried out by the PalEON Project
with support from the National Science Foundation MacroSystems Program
through grants EF-1065702, EF-1065656, DEB-1241874 and DEB-1241868
and from the Notre Dame Environmental Change Initiative. 

Thanks USFS / FIA MOU.

\bibliographystyle{/accounts/gen/vis/paciorek/latex/RSSstylefile/Chicago}

\bibliography{/accounts/gen/vis/paciorek/bibfiles/abbrev.stat,/accounts/gen/vis/paciorek/bibfiles/abbrev.other,/accounts/gen/vis/paciorek/bibfiles/spatstat,/accounts/gen/vis/paciorek/bibfiles/statgeneral,/accounts/gen/vis/paciorek/bibfiles/paciorek,/accounts/gen/vis/paciorek/bibfiles/ecology,/accounts/gen/vis/paciorek/research/jmac/composition_paper/goring_intro}

\end{document}
